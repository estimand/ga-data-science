\documentclass[12pt,aspectratio=169]{beamer}

\usetheme{metropolis}

\definecolor{mDarkBrown}{HTML}{FF5722}
\definecolor{mDarkTeal}{HTML}{263238}
\definecolor{mLightBrown}{HTML}{FF5722}

\usepackage{booktabs}
\usepackage{graphicx}
\usepackage{hyphenat}
\usepackage{multirow}
\usepackage{nicefrac}
\usepackage[normalem]{ulem}

\usepackage[weather]{ifsym}

\usepackage{pifont}
\newcommand{\cmark}{\ding{51}}
\newcommand{\xmark}{\ding{55}}

\usepackage{minted}
\usemintedstyle{tango}
\newminted[bash]{bash}{%
    autogobble,
    bgcolor=mDarkTeal!10,
    linenos
}
\newminted[py3]{python}{%
    python3,
    autogobble,
    bgcolor=mDarkTeal!10,
    linenos
}
\newminted[sql]{sql}{%
    autogobble,
    bgcolor=mDarkTeal!10,
    linenos
}

\usepackage{polyglossia}
\setdefaultlanguage[variant=british]{english}
\usepackage[english=british]{csquotes}

\defaultfontfeatures{Ligatures=TeX}
\setmainfont{Lucida Sans OT}
\setsansfont[Scale=MatchLowercase]{Lucida Sans OT}
\setmonofont[Scale=MatchLowercase]{Lucida Console DK}

\usepackage{mathspec}
\setmathsfont(Digits,Latin,Greek)[Numbers={Lining,Proportional}]{Lucida Bright Math OT}

\newcommand{\mat}[1]{\ensuremath{\mathbf{#1}}}

\newcommand{\R}{\ensuremath{\mathbb{R}}}

\newcommand{\E}[1]{\ensuremath{\mathbb{E}\!\left[ #1 \right]}}
\newcommand{\V}[1]{\ensuremath{\mathbb{V}\!\left[ #1 \right]}}
\newcommand{\Prob}[1]{\ensuremath{\Pr\!\left( #1 \right)}}
\newcommand{\Normal}[2]{\ensuremath{\mathcal{N}\!\left( #1, #2 \right)}}
\newcommand{\simiid}{\ensuremath{\overset{\text{\tiny i.i.d.}}{\sim}}}

\DeclareMathOperator{\logit}{logit}

\author{Gianluca Campanella}
\date{}



\title{Estimation and hypothesis testing}

\begin{document}

\maketitle

\begin{frame}{Contents}
    \tableofcontents[hideallsubsections]
\end{frame}

\begin{frame}{Inference}
    \begin{center}
        {\LARGE%
         \textbf{Desire to generalise}} \\[\medskipamount]
        from a random sample to a population \\
        (from which the sample was selected)
    \end{center}
    \vfill
    \begin{itemize}
        \item Estimation (including uncertainty quantification)
        \item Hypothesis testing
    \end{itemize}
\end{frame}

\section{Estimation}

\begin{frame}{Population and sample}
    \begin{block}{Population}
        The \alert{entire collection of units} possessing one or more
        characteristics we wish to understand (depends on the research question)
    \end{block}
    \vfill
    \begin{block}{Sample}
        A \alert{representative subset} of units for which we collect
        information (known as \alert{observations}) that is then used to
        \alert{estimate} one or more characteristics of the whole population
    \end{block}
\end{frame}

\begin{frame}{Sampling}
    If we draw two samples from the same population, will we always reach the
    same conclusions?
    \vfill\pause
    \begin{center}
        \LARGE%
        \textbf{No!}
    \end{center}
    \begin{itemize}
        \item Sampling variability introduces \alert{uncertainty} in our
              estimates
        \item What happens if we repeat the experiment over and over again?
    \end{itemize}
\end{frame}

\begin{frame}{Estimation}
    \begin{block}{Point estimation}
        One value summarises the characteristic of interest
    \end{block}
    \vfill
    \begin{block}{Interval estimation}
        Two values (an interval), usually together with a point estimate,
        summarise the characteristic of interest and the \alert{uncertainty}
        around the estimate
    \end{block}
\end{frame}

\begin{frame}{Quantifying uncertainty: confidence intervals}
    \begin{itemize}
        \item \alert{Observed} (may change from sample to sample) \\[\bigskipamount]
        \item Defined such that, \alert{were the sampling repeated multiple
              times}, the proportion of CIs that contain the
              population\hyp{}level value would match a certain frequency known
              as confidence level \\
              {\small%
               (Note that there is no such thing as the `probability of
                containing the population\hyp{}level value' within any given
                confidence interval)} \\[\bigskipamount]
        \item 95\% or 99\% confidence levels are typical \\[\bigskipamount]
    \end{itemize}
\end{frame}

\section{Hypothesis testing}

\begin{frame}{Lady tasting tea}
    \only<1>{%
        \begin{block}{Scenario}
            \begin{itemize}
                \item Rothamsted, early 1920s
                \item Given a cup of tea, a lady claims she can tell whether
                      milk or tea was first added to the cup
            \end{itemize}
        \end{block}
        \vfill
        \begin{block}{Question}
            How would you design an experiment to test her claim?
        \end{block}}
    \only<2>{%
        \begin{block}{Scenario}
            \begin{itemize}
                \item To test her claim, Sir Fisher prepares eight cups of tea:
                      \begin{itemize}
                          \item Four have the milk added first
                          \item Four have the tea added first
                      \end{itemize}
                \item The lady performs the experiment by selecting 4 cups \\
                      (e.g.\ those she believes had tea poured first)
            \end{itemize}
        \end{block}
        \vfill
        \begin{block}{Question}
            How many cups does she have to correctly identify to convince
            \alert{you}?
        \end{block}}
    \only<3>{%
        \begin{block}{Questions}
            \begin{itemize}
                \item How many ways are there to choose 4 cups out of 8? \\
                      {\small%
                       (Hint: check \texttt{scipy.misc.comb} or
                        \texttt{sympy.binomial})} \\[\medskipamount]
                \item Of these, how many correspond to correctly identifying\ldots
                      \begin{itemize}
                          \item All 4 cups?
                          \item 3 cups only?
                      \end{itemize}
            \end{itemize}
        \end{block}}
    \only<4->{%
        \begin{block}{Question}
            The lady correctly identifies all 4 cups.
            What can Sir Fisher conclude?
        \end{block}
        \vfill
        \onslide<4>{%
            \begin{itemize}
                \item She has no ability, and has chosen the 4 cups purely by
                      chance
                \item She has the discriminatory ability she claims
            \end{itemize}
            Choosing correctly is unlikely in the first case (1 in 70), so Sir
            Fisher \alert{rejected} this conclusion in favour of the second}}
\end{frame}

\begin{frame}{A/B testing}
    \only<1>{%
        \begin{center}
            \begin{tabular}{lrrr}
                \toprule
                & \multicolumn{2}{c}{Cancelled} & \multicolumn{1}{c}{Total} \\
                \midrule
                Old packaging & 175 & 39.59\% & 442 \\
                New packaging & 168 & 38.27\% & 439 \\
                \bottomrule
            \end{tabular}
            \vfill
            \begin{block}{Question}
                Does the new, nicer, more expensive packaging make customers
                less likely to cancel their subscriptions?
            \end{block}
        \end{center}}
    \only<2>{%
        \begin{center}
            Read the blog post at \\[\medskipamount]
            \url{https://www.candyjapan.com/behind-the-scenes/results-from-box-design-ab-test}
        \end{center}}
\end{frame}

\begin{frame}{Hypothesis testing}
    \begin{enumerate}
        \item Simplify the question into two competing claims:
              \begin{itemize}
                  \item Null hypothesis $H_{0}$
                  \item Alternative hypothesis $H_{1}$ \\[\medskipamount]
              \end{itemize}
        \item Outcome of hypothesis testing is either:
              \begin{itemize}
                  \item `Reject $H_{0}$' (in favour of $H_{1}$)
                  \item `Do not reject $H_{0}$'
              \end{itemize}
    \end{enumerate}
    \vfill
    \begin{itemize}
        \item $H_{0}$ is usually the hypothesis we wish to \alert{disprove}
        \item The test is set up so that it cannot be rejected unless there is
              \alert{sufficient evidence against it}
    \end{itemize}
\end{frame}

\begin{frame}{Absence of evidence is not evidence of absence}
    If we conclude `do not reject $H_{0}$', does it mean $H_{0}$ is true?
    \vfill\pause
    \begin{center}
        \LARGE%
        \textbf{No!}
    \end{center}
    \begin{itemize}
        \item It only means that \alert{there isn't sufficient evidence against
              $H_{0}$}
        \item[$\rightarrow$] The study is inconclusive
    \end{itemize}
\end{frame}

\begin{frame}{Hypothesis testing step-by-step}
    \begin{enumerate}
        \item Choose an appropriate statistical test \\[\medskipamount]
        \item Select a \alert{significance level $\alpha$} \\
              (i.e.\ the probability below which you will reject $H_{0}$) \\[\medskipamount]
        \item Conduct the experiment and record its outcome \\[\medskipamount]
        \item Calculate the \alert{$p$-value} \\
              (i.e.\ the probability of observing something as or more extreme
              than the outcome supposing that $H_{0}$ is true) \\[\medskipamount]
        \item If \alert{$p < \alpha$}, conclude: `$H_{0}$ is rejected at
              significance level $\alpha$' \\
              (the result is `\alert{statistically significant}')
    \end{enumerate}
\end{frame}

\newsavebox{\TP}
\savebox{\TP}{%
    \begin{minipage}[c]{2cm}
        \centering
        \vspace{0.25cm}
        {\LARGE\color{mDarkTeal}\cmark} \\[-0.25cm]
        {\tiny%
         True positive}
        \vspace{0.25cm}
    \end{minipage}}

\newsavebox{\TN}
\savebox{\TN}{%
    \begin{minipage}[c]{2cm}
        \centering
        \vspace{0.25cm}
        {\LARGE\color{mDarkTeal}\cmark} \\[-0.25cm]
        {\tiny%
         True negative}
        \vspace{0.25cm}
    \end{minipage}}

\newsavebox{\FP}
\savebox{\FP}{%
    \begin{minipage}[c]{2cm}
        \centering
        \vspace{0.25cm}
        {\LARGE\color{mDarkBrown}\xmark} \\[-0.25cm]
        {\tiny%
         False positive}
        \vspace{0.25cm}
    \end{minipage}}

\newsavebox{\FN}
\savebox{\FN}{%
    \begin{minipage}[c]{2cm}
        \centering
        \vspace{0.25cm}
        {\LARGE\color{mDarkBrown}\xmark} \\[-0.25cm]
        {\tiny%
         False negative}
        \vspace{0.25cm}
    \end{minipage}}

\begin{frame}[fragile]{What is the significance level $\alpha$?}
    \only<1>{%
        A probability threshold below which:
        \begin{itemize}
            \item The outcome of the test will be deemed `too large' to have
                  occurred under $H_{0}$ (i.e.\ by chance) \\[\medskipamount]
            \item $H_{0}$ will be deemed unlikely given the data
            \item[$\rightarrow$] $H_{0}$ will be rejected
        \end{itemize}}
    \only<2>{%
        \begin{center}
            \begin{tabular}{r|cc}
                                      & \multicolumn{2}{c}{\textbf{State of nature}} \\
                                      & $H_{0}$ is false & $H_{0}$ is true \\ \hline
                Reject $H_{0}$        & \usebox{\TP}     & \usebox{\FP} \\
                Do not reject $H_{0}$ & \usebox{\FN}     & \usebox{\TN} \\
            \end{tabular}
        \end{center}
        \vfill
        \begin{itemize}
            \item[$\rightarrow$] $\alpha$ corresponds to the probability of a
                                 `\alert{type I error}' (false positive) that we
                                 are willing to accept
        \end{itemize}}
\end{frame}

\begin{frame}{Multiple comparisons}
    \begin{block}{Question}
        You are conducting $n$ independent tests at some significance level
        $\alpha$. \\
        What is the probability of at least one false positive finding?
    \end{block}
    \begin{itemize}[<+->]
        \item The probability of a FP in any one test is $\alpha$
        \item The probability of \alert{no} FP in any one test is $1 - \alpha$
        \item The probability of \alert{no} FPs overall is
              $\left( 1 - \alpha \right)^{n}$
        \item The probability of \alert{at least one} FP is
              $1 - \left( 1 - \alpha \right)^{n}$
    \end{itemize}
\end{frame}

\begin{frame}{Multiple comparisons}
    \begin{block}{Question}
        For $\alpha = 5\%$ and $n = 100$ tests, what is the probability of
        $\text{FP} \geq 1$?
    \end{block}
    \vfill\pause
    Using the previous formula\ldots
    \[
        1 - \left( 1 - 0.05 \right)^{100} \approx 0.994 \text{,}
    \]
    which means we are \alert{99.4\% likely to have at least one FP}!
\end{frame}

\begin{frame}{Bonferroni correction}
    \begin{itemize}
        \item Idea: require more evidence to reject $H_{0}$
        \item Using $\alpha^{\prime} = \alpha / n$, the `overall' significance
              level (\alert{family\hyp{}wise error rate}) is approximately
              what we intended
    \end{itemize}
    \vfill
    In the previous example\ldots
    \[
        \alpha^{\prime} = 0.05 / 100 = 0.0005
    \]
    Substituting back\ldots
    \[
        1 - \left( 1 - 0.0005 \right)^{100} \approx 0.05
    \]
\end{frame}

\end{document}

