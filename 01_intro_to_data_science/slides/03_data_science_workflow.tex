\documentclass[12pt,aspectratio=169]{beamer}

\usetheme{metropolis}

\definecolor{mDarkBrown}{HTML}{FF5722}
\definecolor{mDarkTeal}{HTML}{263238}
\definecolor{mLightBrown}{HTML}{FF5722}

\usepackage{booktabs}
\usepackage{graphicx}
\usepackage{hyphenat}
\usepackage{multirow}
\usepackage{nicefrac}
\usepackage[normalem]{ulem}

\usepackage[weather]{ifsym}

\usepackage{pifont}
\newcommand{\cmark}{\ding{51}}
\newcommand{\xmark}{\ding{55}}

\usepackage{minted}
\usemintedstyle{tango}
\newminted[bash]{bash}{%
    autogobble,
    bgcolor=mDarkTeal!10,
    linenos
}
\newminted[py3]{python}{%
    python3,
    autogobble,
    bgcolor=mDarkTeal!10,
    linenos
}
\newminted[sql]{sql}{%
    autogobble,
    bgcolor=mDarkTeal!10,
    linenos
}

\usepackage{polyglossia}
\setdefaultlanguage[variant=british]{english}
\usepackage[english=british]{csquotes}

\defaultfontfeatures{Ligatures=TeX}
\setmainfont{Lucida Sans OT}
\setsansfont[Scale=MatchLowercase]{Lucida Sans OT}
\setmonofont[Scale=MatchLowercase]{Lucida Console DK}

\usepackage{mathspec}
\setmathsfont(Digits,Latin,Greek)[Numbers={Lining,Proportional}]{Lucida Bright Math OT}

\newcommand{\mat}[1]{\ensuremath{\mathbf{#1}}}

\newcommand{\R}{\ensuremath{\mathbb{R}}}

\newcommand{\E}[1]{\ensuremath{\mathbb{E}\!\left[ #1 \right]}}
\newcommand{\V}[1]{\ensuremath{\mathbb{V}\!\left[ #1 \right]}}
\newcommand{\Prob}[1]{\ensuremath{\Pr\!\left( #1 \right)}}
\newcommand{\Normal}[2]{\ensuremath{\mathcal{N}\!\left( #1, #2 \right)}}
\newcommand{\simiid}{\ensuremath{\overset{\text{\tiny i.i.d.}}{\sim}}}

\DeclareMathOperator{\logit}{logit}

\author{Gianluca Campanella}
\date{}



\title{The Data Science workflow}

\begin{document}

\maketitle

\begin{frame}{The Data Science workflow}
    \begin{enumerate}
        \item Define the \alert{research question}
        \item \alert{Get} the data
        \item \alert{Explore} the data
              \begin{itemize}
                  \item (Re)format, clean, merge, stratify\ldots
                  \item Identify trends and outliers
              \end{itemize}
        \item \alert{Model} the data
              \begin{itemize}
                  \item Select and build model(s)
                  \item Evaluate and refine model(s)
              \end{itemize}
        \item \alert{Summarise} the results
              \begin{itemize}
                  \item Condense findings into recommendations
                  \item Describe assumptions and limitations
                  \item Identify follow\hyp{}up research questions
              \end{itemize}
    \end{enumerate}
\end{frame}

\begin{frame}{Time allocation}
    \only<1>{%
        \begin{center}
            \LARGE%
            Which takes longer?
        \end{center}}
    \only<2>{%
        In decreasing order\ldots
        \begin{enumerate}
            \item Defining the problem
            \item Obtaining the data
            \item Cleaning and exploring the data
            \item \alert{Managing expectations}
            \item Summarising the results
            \item \alert{Learning new things}
            \item Modelling
        \end{enumerate}}
\end{frame}

\begin{frame}{The `PR problem' of Data Science}
    Inevitably the data are\ldots\vspace{-1ex}
    \begin{itemize}
        \item Not quite what you need to solve your problem
        \item Too limited, too large, too inaccurate, too expensive to
              obtain\ldots
    \end{itemize}
    \vfill
    But (eventually) you\ldots\vspace{-1ex}
    \begin{itemize}
        \item End up with a `nice' dataset
        \item Apply some models
    \end{itemize}
    \vspace{-1ex}
    \ldots and it \alert{looks} incredibly easy from the outside!
\end{frame}

\begin{frame}[t]{Define the research question}
    \begin{itemize}
        \item Identify the problem and \alert{why} it should be solved
        \item Frame it in the context of data collection
    \end{itemize}
    \vfill
    \begin{block}{Questions to ask}
        \begin{itemize}
            \item Which metric(s) need to be improved?
            \item Which are possible actions to solve the problem?
            \item Which information is necessary and sufficient?
            \item What is the benefit of solving the problem?
        \end{itemize}
    \end{block}
\end{frame}

\begin{frame}[t]{Get the data}
    \begin{itemize}
        \item \alert{Ideal vs available} (`opportunistic' usage)
        \item Limitations
    \end{itemize}
    \vfill
    \begin{block}{Questions to ask}
        \begin{itemize}
            \item Are there enough data?
            \item Are they relevant to the research question?
            \item Can they be trusted?
            \item How were they collected?
        \end{itemize}
    \end{block}
\end{frame}

\begin{frame}[t]{Explore the data}
    \begin{itemize}
        \item Data dictionary and any other documentation
        \item \alert{Descriptive statistics} and \alert{visualisations}
    \end{itemize}
    \vfill
    \begin{block}{Questions to ask}
        \begin{itemize}
            \item What kind of simple visualisations can we use?
            \item Which data types and distributions?
            \item Are there outliers?
            \item Are there missing values?
        \end{itemize}
    \end{block}
\end{frame}

\begin{frame}[t]{Model the data}\vspace{-1ex}
    \begin{itemize}
        \item \alert{Model selection} and fitting
        \item Focus on inference and/or prediction
    \end{itemize}
    \vfill
    \begin{block}{Questions to ask}
        \begin{itemize}
            \item Is there an outcome?
            \item What is an appropriate model for the data?
            \item How can we evaluate model performance?
            \item Can the model be refined?
        \end{itemize}
    \end{block}
\end{frame}

\begin{frame}{Modelling misconceptions}
    Most well\hyp{}executed data science projects don't\ldots\vspace{-1ex}
    \begin{itemize}
        \item Use complicated tools
        \item Fit complicated models
    \end{itemize}
    \vfill
    Instead, they do\ldots\vspace{-1ex}
    \begin{itemize}
        \item \alert{Focus on solving the problem}
        \item Use appropriate --- not necessarily big! --- data
        \item Use relatively standard models
        \item Interpret results sceptically
    \end{itemize}
\end{frame}

\begin{frame}{The 80---20 rule of modelling}
    \begin{itemize}
        \item The first \alert{reasonable} thing you can do goes 80\% of the way
        \item Everything after that is to get the remaining 20\%\ldots \\
              often at additional cost!
    \end{itemize}
    \vfill\pause
    \begin{center}
        \LARGE%
        Is it worth it?
    \end{center}
\end{frame}

\begin{frame}[t]{Summarise the results}
    \begin{itemize}
        \item \alert{Storytelling} and \alert{visual aids} to interpretation
        \item Assumptions and limitations
    \end{itemize}
    \vfill
    \begin{block}{Questions to ask}
        \begin{itemize}
            \item How can I communicate results effectively?
            \item What format should I adopt?
            \item Who are my audience?
            \item How much can I disclose?
        \end{itemize}
    \end{block}
\end{frame}

\begin{frame}{Caveat}
    \begin{center}
        {\Large%
         The Data Science workflow is} \\[\bigskipamount]
        {\LARGE%
         \alert{non\hyp{}linear} and \alert{iterative}} \\[2\bigskipamount]
    \end{center}
\end{frame}

\end{document}

